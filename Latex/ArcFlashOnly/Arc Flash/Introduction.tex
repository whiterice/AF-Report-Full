
%Introduction
\section{Introduction}
\label{af:intro}

\subsection{Introduction to Project}
\label{af:intro:proj}

The Arc Flash Hazard Evaluation Study for \Customer{} was prepared by PowerCore Engineering.\\

The Arc Flash Hazard Evaluation Study determined: the Arcing Fault Clearing Time, Limits of Approach, Working Distance, Incident Energy, Clothing Requirements and Hazard Area Classifications for all electrical equipment within the scope of the study.  Arc Flash Study was performed in compliance with IEEE Std-1584-2002 and CSA Z462-15. The study was based on protective device settings in place at time of inspections.

The study results shall be the basis for implementation of an Arc Flash Hazard Program, that will meet regulation CSA Z462 as well as IEEE-1584.\\
\\
\textbf{Note: as an added benefit, the study shall be compliant with the new NFPA 70E 2015 edition.}\\
\\
OSHA considers CSA Z462 a consensus industry standard for assessing arc flash hazards. Under this legislation, the employer is responsible to assess the hazards in the work place; select, procure and utilize the correct PPE; and document this assessment. OSHA considers Arc Flash assessments that follow CSA Z462, in compliance with OSHA requirements and is accepted practice to protect workers from electrical arc flash hazards.\\


\subsection{Scope}
\label{af:intro:scope}

Our study included the following equipment:
\begin{itemize}
	\item Main Incoming power supply and Associated Equipment.
	\item All 3 Phase Electrical Equipment, excluding the following:
	\begin{itemize}
		\item Equipment supplied by circuit protection rated for less than 60A. 
		\item Equipment below 208V supplied by a transformer rated for less than 125 kVA in its immediate supply.
	\end{itemize}
\end{itemize}
