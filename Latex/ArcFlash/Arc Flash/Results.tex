%Arc Flash Study Results & Recommendations

\section{Results}
\label{af:results}

\subsection{Arc Flash Study Results and Recommendations}
\label{af:results:afrr}

The Arc Flash Hazard Evaluation Study has shown that:
\begin{enumerate}
\item WDMB01 is Arc Hazard Class 4
\item WDMB02-SW is Arc Hazard Class 4
\item WDMB03-SW is Arc Hazard Class 4
\item WDMB04-SW is Arc Hazard Class 4
\item WDMB05-SW is Arc Hazard Class 4
\item WDMB06-SW is Arc Hazard Class 4
\item WDMB07-SW is Arc Hazard Class 4
\item WDMB08-SW is Arc Hazard Class 4
\item WDMDP01 is Arc Hazard Class 4

\end{enumerate}

\pagebreak

\subsection{Arc Flash Study Details}
\label{af:results:afsd}

The following issues have been encountered and are of note:

\begin{itemize} 
\item The risk category for all downstream equipment from the last bus or power panel listed on the summary sheet and shown on the drawings shall be treated as the last indicated Hazard Category, unless indicated otherwise.

\item Arc Flash Hazard Identification labels for all equipment within the scope of this study will be affixed in a visible area.

\item Section 4.3.5.1 of the CSA Z462 standard states. "An arc flash hazard analysis shall determine the (a) arc flash boundary; (b) incident energy at the working distance; and (c) PPE that personnel within the arc flash boundary shall use.\cite{CSA}

\item Modifications of the equipment, changes to system configuration, adjustment of protective device settings or failure to properly maintain equipment may invalidate the study results. Therefore the analysis shall be updated when a major modification or renovation takes place. It shall also be reviewed periodically, not to exceed 5 years, to account for changes in the electrical distribution system that could affect the results of the analysis.\cite{CSA}

\item The analysis assumes all circuit protective devices are operational and regular maintenance testing/inspection is performed.  Non-functioning overcurrent devices can allow arcing faults to persist for much longer than normal, potentially presenting more dangerous arc flash hazards than stipulated in this study.

\item Arc flash study results and personal protective equipment (PPE) levels are no substitute for safe work practices.  As stated in CSA-Z462-15 4.3.7.3.16 "'While some situations could result in burns to the skin, even with the protection specified in Table 5, burn injury will likely be reduced and be survivable. Due to the explosive effect of some arc events, physical trauma injuries can occur. The requirements of this Clause do not provide protection against physical trauma other than exposure to the thermal effects of an arc flash."' \cite{CSA} Always remember PPE is not intended to prevent all injuries, but to mitigate the impact of an arc flash on an individual.
 
\item\textbf{Equipment below 240 V need not be considered (for Arc Flash Hazard Analysis) unless it involves at least one 125 kVA or larger low impedance transformer in its immediate power supply.}\cite{IEEE}	

\item Available short circuit current provided by the Local Distribution Company (LDC) did not include an X/R ratio or circuit protection information as this can change without notice. As a result, a conservative X/R ration of 15 and a maximum clearing time from the LDC circuit protection was used for this study. 
\end{itemize}
\pagebreak
\subsection{Study Recommendations}
\label{af:results:afsr}

Based on the Arc Flash evaluation Study performed, we recommend the following:
\begin{itemize}
\item We recommend that all personnel authorized to operate any electrical equipment be properly educated on Arc Flash Hazard and trained  in use of arc flash PPE (Personal Protective Equipment)

\item	Provide PPE of appropriate level to all personnel that will be operating electrical equipment.

\item	We recommend that a formal written Electrical Equipment Operation Procedures Manual as well as an Arc Flash Hazard Program that meets the regulations noted in CSA Z462-15 and IEEE-1584, be developed and implemented.

\item All modifications and recommendations in the enclosed \textbf{Recommended Changes Summary} list should be reviewed and implemented as necessary.

\end{itemize}
\vspace{10mm}
\noindent Thank you for this opportunity to be of service to you.  If you have any questions regarding the recommendations in this report or any other matter, please contact our London Engineering Services office at (519) 474-1175. \newline
\vspace{5mm}
\\
\noindent Sincerely,\newline

\vspace{5mm}
\noindent\textbf{PowerCore Engineering}\newline

%Scott Signature
%\begin{comment}
\begin{multicols}{2}
\centering
\includegraphics[height=0.5in, keepaspectratio=true]{../Images/Roman_signature.jpg} \\
Roman Bulla, P. Eng. \\Power Systems Engineer \\
\includegraphics[height=0.5in, keepaspectratio=true]{../Images/Scott_signature.jpg} \\
Scott Vermeire, P. Eng. \\Power Systems Engineer \\
\end{multicols}
%\end{comment}

%Vince Signature
\begin{comment}
\begin{multicols}{2}
\centering
\includegraphics[height=0.5in, keepaspectratio=true]{../Images/Roman_signature.jpg} \\
Roman Bulla, P. Eng. \\Power Systems Engineer \\
\includegraphics[height=0.5in, keepaspectratio=true]{../Images/Vince_signature.jpg} \\
Vince Klingenberger \\Electrical Engineering Technologist \\
\end{multicols}
\end{comment}