%Objectives
\section{Objectives}
\label{af:o}

\subsection{Protective Device Coordination Objectives}
\label{af:o:afso}

The purpose of a Coordination Study is to select settings and characteristics for the
Protective Devices in order to achieve maximum selectivity between devices during
fault conditions. With complete selectivity, the device nearest the fault will operate first,
thus minimising the interruption. However, due to the characteristics of various devices,
complete selectivity cannot always be achieved without some loss of protection,
therefore some compromises must be considered.\\

A complete Coordination Study is a step by step evaluation of the total plant power
distribution system in which this delicate balance between protection and coordination is
determined with the aid of time vs. current curves of the various devices.\\

The first step in conducting a Coordination Study is a Fault Analysis or Short Circuit
Study. The purpose of a Short Circuit Study is to determine the fault current magnitude
at critical points in the power system under worst conditions (i.e. bolted three phase
fault). The calculated fault levels are then used to determine the acceptability of the
protective devices.