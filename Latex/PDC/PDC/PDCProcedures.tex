%Procedures
\section{Protective Device Coordination Study Procedures}
\label{af:pdcp}

\noindent The time-current characteristics of the system fuses and low voltage breakers have been computer generated and are graphically displayed on log-log paper as time
versus current magnitudes. In developing the protective device settings, consideration was given to both isolation of faults and protection of equipment such as cable, motors, and transformers. Minimum requirements for equipment protection as outlined in the Canadian Electrical Code and publications of the Canadian Standards Association were followed in all cases.\\

\noindent Referring to the coordination curves, low voltage breaker instantaneous trip curves are ended at the maximum symmetrical three-phase fault current for the bus on which the device is applied. These values were obtained from the Short Circuit Study. Time delay relay and breaker short delay trip curves are ended at the maximum symmetrical three phase fault current determined with all motor contributions omitted. All protective devices were selected and applied in accordance with all appropriate standards.\\

\noindent Also shown on the coordination curves are the transformers through fault withstand damage curves, as detailed in CSA Standards. These curves represent the mechanical and thermal withstand limits of the transformer. Two withstand damage curves are plotted for delta-wye solidly grounded transformers. These are single-phase line to ground and three phase secondary faults. The singlephase line to ground damage curve is omitted on ungrounded or impedance grounded transformers. The transformer primary main protective device should interrupt the fault before these time limits are reached.\\

\noindent Transformer inrush current is represented by a point located at 0.1 seconds and 12 times transformer full load amps. The primary protective device must be set above
inrush current to prevent nuisance tripping upon energising the transformer. It must also be set above the maximum asymmetrical fault at the secondary voltage to prevent operation for secondary faults. The overcurrent relay of transformer feeder breakers has an instantaneous element setting based on the following:\\

\[INST = (1.1)(Isym)(A.F. )/(TXMR - Ratio)(CT - Ratio)\]
	
\noindent where:

\begin{itemize}
	\item \emph{Isym} is the maximum three phase momentary symmetrical fault current flowing through the relay.
	\item \emph{A.F.} is the asymmetry factor multiplier for maximum single phase RMS current at half-cycle from table 61 pg. 241 of IEEE STD 241 - 1974. This multiplier is determined by the X/R ratio of the faulted bus.
	\item \emph{CT-Ratio} is the current transformer ratio.
	\item \emph{TXMR-Ratio} is the transformer voltage ratio.
\end{itemize}

\noindent Cable thermal damage limits, as published by the IPCEA, have been shown where appropriate. These damage limits represent the maximum time/current capabilities of
the cable insulation for through faults. The branch protective device should operate before this limit is exceeded.\\

\noindent When the system contains more than one voltage level, the individual protective device characteristics are normalised to a common voltage base. This voltage base, as well as current scale, is noted at the top and bottom of the curves.\\

\noindent \textit{The recommended settings must be implemented to assure proper selectivity of the protective devices.}
